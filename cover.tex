\title{Mining the human microbiome for clinical insight}

\author{Claire Marie No{\"e}lle Duvallet}

\prevdegrees{B.S., Columbia University (2013)}

\department{Department of Biological Engineering}

\degree{Doctor of Philosophy in Biological Engineering}

% Valid degree months are September,
% February, or June.  The default is June.
\degreemonth{June}
\degreeyear{2019}
\thesisdate{11 January 2019}

%% By default, the thesis will be copyrighted to MIT.  If you need to copyright
%% the thesis to yourself, just specify the `vi' documentclass option.  If for
%% some reason you want to exactly specify the copyright notice text, you can
%% use the \copyrightnoticetext command.
%\copyrightnoticetext{\copyright IBM, 1990.  Do not open till Xmas.}

% If there is more than one supervisor, use the \supervisor command
% once for each.
\supervisor{Eric J. Alm}{Professor}

% This is the department committee chairman, not the thesis committee
% chairman.  You should replace this with your Department's Committee
% Chairman.
\chairman{Forest White}{Chair of Graduate Program, Department of Biological Engineering}

% Make the titlepage based on the above information.  If you need
% something special and can't use the standard form, you can specify
% the exact text of the titlepage yourself.  Put it in a titlepage
% environment and leave blank lines where you want vertical space.
% The spaces will be adjusted to fill the entire page.  The dotted
% lines for the signatures are made with the \signature command.
\maketitle

% The abstractpage environment sets up everything on the page except
% the text itself.  The title and other header material are put at the
% top of the page, and the supervisors are listed at the bottom.  A
% new page is begun both before and after.  Of course, an abstract may
% be more than one page itself.  If you need more control over the
% format of the page, you can use the abstract environment, which puts
% the word "Abstract" at the beginning and single spaces its text.

% You can either \input (*not* \include) your abstract file, or you can put
% the text of the abstract directly between the \begin{abstractpage} and
% \end{abstractpage} commands.
\cleardoublepage
\setcounter{savepage}{\thepage}
\begin{abstractpage}
%% The text of your abstract and nothing else (other than comments) goes here.
%% It will be single-spaced and the rest of the text that is supposed to go on
%% the abstract page will be generated by the abstractpage environment.

The human microbiome is essential for health and has been implicated in many diseases.
DNA sequencing has enabled the detailed characterization of these human-associated microbial communities, leading to a rapid expansion in studies investigating the human microbiome.
%However, extracting clinically-relevant associations from microbiome datasets remains challenging because of the high dimensional nature of the data and variability across studies.
In this thesis, I describe three projects which overcome various data analysis challenges to extract useful clinical insights from microbiome data.
In the first project, I present an analysis of lung, stomach, and oropharyngeal microbiomes.
I leverage data collected from multiple sites per patient to identify aspiration-associated changes in the relationships between aerodigestive communities, discovering new properties of the aerodigestive microbiome and suggesting new approaches for treatment.
%These changes suggest new approaches for developing diagnostics and treatments for aspiration-related respiratory complications.
%I leverage data collected from multiple sites per patient to find aspiration-associated changes in the relationships between aerodigestive communities which suggests new targets for treatment and diagnosis.
In the second project, I perform a meta-analysis of case-control gut microbiome datasets with standard data processing and analysis methods.
I find consistent patterns characterizing disease-associated microbiome changes and a set of shared associations which could inform clinical treatment and therapeutic development approaches for many different microbiome-mediate diseases.
In the third project, I describe a framework for rational donor selection in fecal microbiota transplant clinical trials.
In this framework, knowledge derived from clinical and basic science research is used to inform which donor is selected for fecal transplants, increasing the likelihood of successful trials.
Together, these projects demonstrate a variety of approaches to mine the human microbiome for clinically-relevant insights and suggests multiple avenues forward for translating findings from microbiome data analyses into clinical impact.

\end{abstractpage}

\cleardoublepage
