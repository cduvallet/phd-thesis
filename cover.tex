\title{Mining the human microbiome for clinical insight}

\author{Claire Marie No{\"e}lle Duvallet}

\prevdegrees{B.S., Columbia University (2013)}

\department{Department of Biological Engineering}

\degree{Doctor of Philosophy in Biological Engineering}

% Valid degree months are September,
% February, or June.  The default is June.
\degreemonth{June}
\degreeyear{2019}
\thesisdate{11 January 2019}

%% By default, the thesis will be copyrighted to MIT.  If you need to copyright
%% the thesis to yourself, just specify the `vi' documentclass option.  If for
%% some reason you want to exactly specify the copyright notice text, you can
%% use the \copyrightnoticetext command.
%\copyrightnoticetext{\copyright IBM, 1990.  Do not open till Xmas.}

% If there is more than one supervisor, use the \supervisor command
% once for each.
\supervisor{Eric J. Alm}{Professor}

% This is the department committee chairman, not the thesis committee
% chairman.  You should replace this with your Department's Committee
% Chairman.
\chairman{Forest White}{Chair of Graduate Program, Department of Biological Engineering}

% Make the titlepage based on the above information.  If you need
% something special and can't use the standard form, you can specify
% the exact text of the titlepage yourself.  Put it in a titlepage
% environment and leave blank lines where you want vertical space.
% The spaces will be adjusted to fill the entire page.  The dotted
% lines for the signatures are made with the \signature command.
\maketitle

% The abstractpage environment sets up everything on the page except
% the text itself.  The title and other header material are put at the
% top of the page, and the supervisors are listed at the bottom.  A
% new page is begun both before and after.  Of course, an abstract may
% be more than one page itself.  If you need more control over the
% format of the page, you can use the abstract environment, which puts
% the word "Abstract" at the beginning and single spaces its text.

% You can either \input (*not* \include) your abstract file, or you can put
% the text of the abstract directly between the \begin{abstractpage} and
% \end{abstractpage} commands.
\cleardoublepage
\setcounter{savepage}{\thepage}
\begin{abstractpage}
%% The text of your abstract and nothing else (other than comments) goes here.
%% It will be single-spaced and the rest of the text that is supposed to go on
%% the abstract page will be generated by the abstractpage environment.

Human-associated microbial communities are essential for health and have been implicated in many diseases.
DNA sequencing technology has enabled the detailed characterization of these communities, leading to a rapid expansion in studies investigating relationships between the human microbiome and disease.
However, identifying clinically-relevant associations from microbiome datasets is complicated by the high dimensional nature of the data and variability of communities across people.
In this thesis, I describe three projects which overcome various analysis challenges to identify clinically-relevant associations between the human microbiome and disease.
In the first project, I present an analysis of lung, stomach, and oropharyngeal microbiomes of pediatric patients with aerodigestive symptoms.
I leverage data collected from multiple sites per patient to identify  clinically-actionable alterations in aerodigestive community relationships in patients with swallowing dysfunction and to discover new characteristics of the human lung and stomach microbiomes.
In the second project, I perform a meta-analysis of case-control gut microbiome datasets across 28 studies and 10 diseases.
By standardizing processing and analysis methods across many datasets, I find consistent disease-specific and shared patterns of associations which can inform therapeutic development and clinical treatment approaches.
In the third project, I describe a framework for performing rational donor selection in fecal microbiota transplant clinical trials.
In this framework, microbial associations identified from clinical and basic research are used to inform the donor selection strategy, increasing the likelihood of a successful clinical trial.
These projects demonstrate a variety of approaches for mining human microbiome data to identify clinically-relevant associations and discover new fundamental properities of human-associated microbial communities.
Together, this work suggests multiple avenues forward for translating findings from microbiome data analyses into clinical impact.

\end{abstractpage}

\cleardoublepage
