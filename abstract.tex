%% The text of your abstract and nothing else (other than comments) goes here.
%% It will be single-spaced and the rest of the text that is supposed to go on
%% the abstract page will be generated by the abstractpage environment.

Human-associated microbial communities are essential for health and have been implicated in many diseases.
DNA sequencing technology has enabled the detailed characterization of these communities, leading to a rapid expansion in studies investigating relationships between the human microbiome and disease.
However, identifying clinically-relevant associations from microbiome datasets is complicated by the high dimensional nature of the data and variability of communities across people.
In this thesis, I describe three projects which overcome various analysis challenges to identify clinically-relevant associations between the human microbiome and disease.
In the first project, I present an analysis of lung, stomach, and oropharyngeal microbiomes of pediatric patients with aerodigestive symptoms.
I leverage data collected from multiple sites per patient to identify  clinically-actionable alterations in aerodigestive community relationships in patients with swallowing dysfunction and to discover new characteristics of the human lung and stomach microbiomes.
In the second project, I perform a meta-analysis of case-control gut microbiome datasets across 28 studies and 10 diseases.
By standardizing processing and analysis methods across many datasets, I find consistent disease-specific and shared patterns of associations which can inform therapeutic development and clinical treatment approaches.
In the third project, I describe a framework for performing rational donor selection in fecal microbiota transplant clinical trials.
In this framework, microbial associations identified from clinical and basic research are used to inform the donor selection strategy, increasing the likelihood of a successful clinical trial.
These projects demonstrate a variety of approaches for mining human microbiome data to identify clinically-relevant associations and discover new fundamental properities of human-associated microbial communities.
Together, this work suggests multiple avenues forward for translating findings from microbiome data analyses into clinical impact.
