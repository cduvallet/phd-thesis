\section*{Acknowledgments}

\begin{singlespace}

I look back on my PhD with almost exclusive positivity: the road was rough at times, but graduate school was overall an overwhelmingly enriching and positive experience.
I recognize that this is a rare sentiment to have upon graduation, and I credit my communities, role models, mentors, family, and friends for this seemingly impossible feat.

If I have thrived in grad school, it is in large part because of my advisor, Eric Alm.
He wayers whatever seeds his students decide to plant, and I am deeply grateful for him giving me the support I needed in my research and the space I needed to succeed in my non-scientific endeavors.
Learning how to do science from Eric has given me the confidence to tackle hard problems, jump into creative solutions, and figure out the details later.
Thanks, above all, for believing in me and reminding me of that often enough for it to stick.

I am also grateful to my committee, who was a constant source of support and never one of stress.
Rafael Irizarry welcomed me into his group and gave his mentorship freely, even when my projects did not fall within his direct expertise.
I've learned so much from his perspectives on statistics and data science, which I will keep beyond my PhD.
I hope to always remain a friend of the lab.
Deborah Hung provided a much-appreciated clinical and public health perspective to my work and career path.
And finally thanks to my committee chair, Jim Collins, for providing realism and unequivocal support throughout my PhD.

Rachel Rosen was the clinical PI on the aerodigestive microbiome project, and I thank her for sharing her data and knowledge and, most importantly, her passion for her patients.
Working with her -- a real doctor! -- was an incredible privilege, and provided me with important perspectives on the realities of doing actual translational research.
Elizabeth Kujawinski was a mentor and colleague on the sewage project, and taught me basically everything I know about metabolomics. I thank her for her generosity and for infecting me with her passion for chemistry.

There are many postdocs and senior graduate students whom I credit for putting me onto a path to success.
Ilana Brito was the mentor I didn't know I needed, and inspired me to hustle through this last year of my PhD to find my way to (and hopefully through) an untraditional career path.
Sarah Spencer also provided key ``unsolicited advice'' early in my rotation and thesis proposal processes which was incredibly helpful.
Her laughter is infectious and her legacy of generosity in the lab is one I aspire to.
Thomas Gurry was my first introduction to computational work during my rotation, and gave me much more time and energy than was required of him.
He provided me a solid computational foundation and was the first to introduce me to the possibility of a PhD with only computational work, which would become my passion.
In my first years in the lab, I worked closely with Mariana Matus on the Underworlds sewage project, which was key to helping me identify public health as a field where my passions and skills intersect.
Learning from her entrepreneurial spirit convinced me that there are no limits to what can be done in graduate school.
Scott Olesen was many things to me throughout my PhD: a role model, mentor, collaborator, and friend.
He was a crucially positive force in almost every aspect of my PhD: starting me on my path to student advocacy, sharing tips and tricks (and rants!) about computing, and being a model of a grad student to aspire to.
There are few people in my life who have had an impact like Scott's on shaping who I am.
Finally, I send infinite thanks to Sean Gibbons, my close collaborator on the meta-analysis and batch correction projects.
Sean is such a joy to work with, so purely believes in and loves science, and re-inspired me through my existential crises when I stopped thinking that data meant anything at all.
Beyond being just a mentor or role model, Sean is a true sponsor: volunteering me for important opportunities and always at the ready to boost my scientific career.
If more scientists (and people!) were like Sean, the world would be a better place.

I am deeply grateful to the entire Alm lab for being a community of laughter, curiosity, and generosity.
The past few years have been a joy and it is in large part because of the culture that is cultivated and sustained in the group.
I especially could not imagine my PhD without Nathaniel Chu, who was an amazing office mate and provided endless patience and insights for our many discussions ranging across many topics -- scientific and otherwise.
Thanks also for the infinite guacamole and all the M\&M's.
Mathieu Groussin was also an excellent compatriot in Office 307 -- thanks for the jokes, wisdom, and excellent haircuts.
Finally, no one in the Alm lab has any chance at success without Shandrina Burns and I thank her for her flexibility, support, and patience. She truly works magic.

I was also happy to become a friend of the Rafalab, and I thank the entire group for welcoming me and for giving me access to a community where computationalists are the norm.
Though they may not realize it, Stephanie Hicks and Keegan Korthauer are important role models of mine, as thoughtful and positive people who stick up for what's right and strong women who do computational work.
Along with Patrick Kimes and Alejandro Reyes, I also thank them for humoring my questions about R and patiently teaching me new things in spite of my commitment to Python.

Beyond the work presented in these pages, I spent a significant portion of my PhD in student leadership in my department and at MIT.
Learning conflict management skills as a BE REF positively impacted almost every aspect of my life.
This impact wouldn't have happened without the people involved in my early REFS days: Libby Mahaffy, Scott, Georgia Lagoudas, Deena Rennerfeldt, Tu Nguyen, and Souparno Ghosh.
I also thank my early diversity co-conspirators, Scott Olesen and Manu Kumar, for sharing in my rants and helping me channel my passions toward productive change.
I was able to grow tremendously as a student leader because of our receptive departmental leadership, and I am very thankful to Doug Lauffenberger for being thoughtful, supportive, and responsive to our efforts.
My current GSC diversity crew, Nasir Almasri, Josue Lopez, Danielle Cosio, German Parada, Halston Lim, and Bianca Lepe, revitalized me in this work when I was closest to burning out.
The future of diversity and equity at MIT is bright if they are leaders in it.
Similarly, I am grateful for the brief time I overlapped with Jess Boles and Parker Vascik in GradSAGE, and am excited to watch them positively transform the culture of advising at MIT.

Finally, I thank my friends, family, and frisbee teammates for keeping me happy and healthy during my PhD.
I cannot thank Carolyn Saund and Megan Armstrong enough for being right by my side these past years, for supporting my growth, and for growing alongside me as we figure out how to maneuver this world.
Hold on tight to your passion, and thanks in advance for reminding me to do the same.
When I think about strong support networks of badass women, Andee Wallace and Shelby Doyle also come to mind.
Maneuvering grad school together enriched my experience tremendously, kept me focused on what matters, and was a lot of fun.
Thanks also to Brett Geiger, the BE 2014 cohort, and the entire BE community for being great friends and colleagues.
I am also thankful to Janyne Little, who keeps me adventuring and reminds me about the world outside my urban ivory towers.
Jeremy Pivor moving back to Boston was pivotal for my happiness in this city: I am so lucky for his friendship and treasure his radiant soul.
I've also been lucky to share living spaces with incredibly supportive and friendly roommates: Kunal Bhutani, a man kinder and more thoughtful than I will ever be, and the 9 Seattleites, including my dear childhood friend (and yet another badass woman inspiration) Kelly Moynihan, as well as Khoi Nguyen, Andee, Ashvin Bashyam, Griffin Clausen, Santiago Correa-Echevarria, Shawn Musgrave, and of course Nancy, Barb, Sonia, and Ruth.

To everyone I have ever thrown a frisbee with: thank you for keeping me physically and mentally healthy. I am especially grateful to have been a member of sMITe, MIT's women's ultimate frisbee team, during my entire PhD. I treasure having had the opportunity to learn about MIT's fascinating undergraduate culture and to witness the transformation of so many young women into confident players and leaders. Lisa, Bethany, Margalit, Oreo, Kimmy, Kelly, Michelle, Mary, Amy, Theresa, Sam, Ivana, and everyone else on sMITe: thanks for putting up with me, and for being my friend.

I also thank my family for supporting and encouraging me my entire life. Felix for encouraging me to adopt life-changing habits like Getting Things Done and meditation; Etienne, for putting us to shame by having a real job and providing important perspective which grounded my research on IBD; and Maman et Papa, for always supporting me, recognizing my passions better than I do, and putting up with me talking about poop over many many dinners.

\end{singlespace}
