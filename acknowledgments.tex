\section*{Acknowledgments}

A PhD is not an individual endeavor, and I owe my success to the support and encouragement of many people.

First, to my advisor Eric Alm.
If I have thrived in grad school, it is in large part because of you: your mentorship, your approach to graduate school training, and the lab culture you have cultivated.
Thank you for being the kind of advisor who waters whatever seeds his students plant, for giving me the support I needed in my research and the space I needed to succeed in my non-scientific endeavors.
I am deeply appreciative of your willingness to share the philosophies underlying your actions, to admit and correct for mistakes, and to trust us with as much responsibility as we want to take.
Paired with your humor and inexhaustible childlike excitement for good science and cool data, this has been an amazingly positive journey in which I have learned and grown more than I would have ever expected.
Watching you do science and learning from it has given me the confidence (and dare I say arrogance) to tackle hard problems, jump into creative solutions, and figure out the details later.
Thanks, above all, for believing in me and reminding me of that often enough for it to stick.

To my committee, thank you for being a constant source of support and never one of stress.
To Rafael Irizarry: thank you for so generously welcoming me into your group and for your mentorship, even though my projects did not fall within your direct expertise.
I deeply value having learned about your perspectives on statistics and data science, and will take them with me beyond my PhD.
I hope to always remain a friend of the lab.
To Deborah Hung, thank you for beginning every single meeting we had with an enthusiastic ``how can I help?'' and for providing a much-appreciated clinical and public health perspective to my work and career path.
To my committee chair, Jim Collins: thank you also for your realism and your unending support.

To Rachel Rosen, clinical lead on the aerodigestive microbiome project and a close collaborator: thank you for the opportunity to work on this project, and for sharing your passion for your patients with me.
I feel so lucky to have had the opportunity to work so closely with you - a real doctor! - and take away important perspectives on the realities of doing real translational research from our work together.
To Elizabeth Kujawinski, mentor and colleague on the sewage project: thank you for teaching me everything I know about metabolomics, and for infecting me with your passion for chemistry.
If I had to re-do my PhD, I would probably choose metabolomics because of our work together.

To the postdocs and senior graduate students who shaped my early PhD: thank you for sharing your time and wisdom, and for setting me on a path to success.
To Ilana Brito, thank you for being the mentor I didn't know I needed and for inspiring me to hustle through my last year of my PhD and find my way through an uncertain career path.
Thanks to Thomas Gurry, my first introduction to computational work during my rotation in the Alm lab, for your patience and incredible generosity, teaching me so much about the basics of computing and enabling me to find my passion here.
Mariana Matus, watching you maneuver your PhD with your entrepreneurial spirit showed me that there are no limits to what can be done in graduate school.
Thank you also for welcoming me into the Underworlds project, opening my eyes to the possibilities in sewage and helping me identify public health as a place where my passions and skills intersect.
To Scott Olesen, a role model, mentor, collaborator, and friend: you were a crucially positive force in almost every aspect of my PhD, from REFS and diversity to computational work and statistics.
Thank you for starting me on my path to student advocacy, for sharing your tips and tricks (and rants!) about computing, and for being a model of a grad student to aspire to.
Finally, to Sean Gibbons, my close collaborator on the meta-analysis and batch correction work: thank you for being a joy to work with, for so purely believing in and loving science, and for re-inspiring me through my existential crises when I stopped thinking that data meant anything at all.
If more scientists (and people!) were like you, the world would be a better place.

To the entire Alm lab, thank you for being a community of laughter, curiosity, and generosity.
To Nathaniel Chu especially: I could not imagine a PhD without you as my office mate.
Thank you for your patience and your insights in our discussions ranging across visualizations, papers, and mentorship.
Thanks also for the infinite guacamole and all the M&M's.
To Mathieu Groussin as well, thank you for your jokes, wisdom, and knowledge of phylogenetic trees.
Finally, to Shandrina Burns: thank you for your flexibility and support.
You truly work magic, and no one in the Alm lab would succeed without you.

To the members of the rafalab, thank you for welcoming me into your group and for giving me access to a community where computationalists are the norm, not the exception.
To Stephanie Hicks and Keegan Korthauer especially: thank you for your thoughtfulness and positivity in all that you do, and for publicly modeling sticking up for what's right.
Having you as strong women to look up to as computational mentors has meant so much to me.
Along with Patrick Kimes and Alejandro Reyes, thank you also for humoring my questions about R and patiently teaching me new things in spite of my commitment to Python.

Beyond the work presented in these pages, I also spent a significant portion of my PhD in student leadership positives in my department and at MIT.
Learning conflict management skills as a BE REF positively impacted almost every aspect of my life.
This change wouldn't have been possible without the people involved in my early REFS days: Libby Mahaffy, Scott Olesen, Georgia Lagoudas, Deena Rennerfeldt, Tu Nguyen, and Souparno Ghosh.
To my diversity co-conspirators, Scott Olesen and Manu Kumar: thank you for listening and sharing in my rants and for helping me channel my passions toward productive change.
To the current GSC crew, Nasir Almasri, Josue Lopex, Danielle Cosio, German Parada, Halston Lim, and Bianca Lepe: thank you for revitalizing me in this work.
The future of diversity and equity at MIT is bright if you are leaders in it.

Finally, to my friends, family, and frisbee teammates: thank you for keeping me happy and healthy these past years.
Carolyn Saund and Megan Armstrong: I can't thank you enough for being with me these years, for supporting my growth and for growing alongside me as we figure out how to maneuver this world.
Hold on tight to your passion, and thanks in advance for reminding me to do the same.
To Andee Wallace and Shelby Doyle: doing grad school has been so much more palatable because we've done it together.
When I think about strong support networks of badass women, I think of you.


when I think about women supporting women and building strong support networks, I think of you two.


thanks for  it's meant so much to do this together with you



Friends, family, and frisbee
Carolyn and Megan
Andee and Shelby
Roommates
Janyne for keeping me adventuring
sMITe
Felix (for GTD, 20% rule, and inspiration to version control everything)
ET (for putting us to shame by having a job and real life)
Parents (for supporting me, recognizing my passions better than I do, and listening to me talk about poop over many dinners)


I am so happy that you are

thank you for so much more

Thomas - being my first introduction to computational work with patience and enthusiasm, enabling me to find my passion here
Mariana - drive and ability to make impossible things actually happen was an inspiration, expanded the scope of what a PhD student could do
Scott - being a role model, mentor, collaborator, and now close friend. Hope you remain all four forever. Also for encouraging me to start doing diversity and REFS
Sean Gibbons - modeling pure interest in science for the sake of science, encouraging me and being patient when I thought it was all fake, modeling success and being a joy to work with, constant sponsor of my work (opened a really important door into the world of open-source software)


and Elizabeth Kujawinski




Some of my proudest PhD moments were answering your statistics questions correctly in your group meeting and being named a friend of the lab, which I will always remain.

Your
I hope to take these attributes into any future relationships I find myself on the advisor side of.

transparency in decision-making, your willingess
I deeply appreciate and respect your willingness to share the underlying philosophies motivating your actions, and hope to approach any future advisees of mine with a similar commitment to transparency.



thank you for your encouragement
Because of your mentorship - both direct and by example - and the lab culture that you've created, I have grown tremendously these past years, technically and personally.

Your commitment to science, motivated by discovery and always infused with childlike excitement at new ideas, and



When I joined the Alm lab, I saw a place where the seeds placed

Thank you for creating an environment in which our growth, both personal and technical, was encouraged and watered until we thrived.

Your approach to science, motivated by discovery and

Your approach to science, creative

commitment to and willingness

Your commitment to your trainees' success

Mentors:
Rafa - will always be a friend of the lab
Jim
Deb

Ilana - for being the mentor I didn't know I needed
Liz - teaching me everything I know about metabolomics and being so giving with time
Rachel

Thomas - being my first introduction to computational work with patience and enthusiasm, enabling me to find my passion here
Mariana - drive and ability to make impossible things actually happen was an inspiration, expanded the scope of what a PhD student could do
Scott - being a role model, mentor, collaborator, and now close friend. Hope you remain all four forever. Also for encouraging me to start doing diversity and REFS
Sean Gibbons - modeling pure interest in science for the sake of science, encouraging me and being patient when I thought it was all fake, modeling success and being a joy to work with, constant sponsor of my work (opened a really important door into the world of open-source software)

Stephanie, Keegan - inspirational strong women doing good science and sticking up for what's right
Patrick and Alejandro, for humoring my questions about R and patiently teaching me new things despite my commitment to Python.


Shandrina

Peers:
Nathaniel
Diversity co-conspirators: Scott and Manu, Bevin, and now my GSC crew for revitalizing me in the work (Nas, Josue, Danielle, German, Halston, and Bianca)

Funding:
NDSEG
Siebel

Friends, family, and frisbee
Carolyn and Megan
Andee and Shelby
Janyne for keeping me adventuring
sMITe
Felix (for GTD, 20% rule, and inspiration to version control everything)
ET (for putting us to shame by having a job and real life)
Parents (for supporting me, recognizing my passions better than I do, and listening to me talk about poop over many dinners)
