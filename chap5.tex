%% This is an example first chapter.  You should put chapter/appendix that you
%% write into a separate file, and add a line \include{yourfilename} to
%% main.tex, where `yourfilename.tex' is the name of the chapter/appendix file.
%% You can process specific files by typing their names in at the
%% \files=
%% prompt when you run the file main.tex through LaTeX.
\chapter{Conclusions}

\section{Summary and limitations of reported work}

In this thesis, I present three projects which overcome a variety of challenges to mine human microbiome data to extract clinical insights.
In the first project, I overcome the difficulty of identifying consistent biomarkers across highly variable lung and stomach communities by instead looking at the relationships between aerodigestive body sites within individual patients.
In the second project, I re-analyze gut microbiome datasets across many diseases and individual studies to identify consistent associations despite the technical challenges involved in extracting generalizable knowledge from the existing corpus of published studies.
In the third project, I present a framework to categorize disease-specific insights gleaned from previously performed analyses and experiments and leverage them to improve fecal microbiota transplant clinical trials.
Although performed in different contexts and with different goals in mind, these three computational projects share limitations which suggest further work to test and validate their findings.

The major limitation of this thesis is that the findings in each project suggest associations but do not and cannot confirm causal relationships.
In Chapter 2, although I demonstrate that the interpretation of the aspiration-associated perturbation that I find is consistent across a variety of analyses and metrics, these associations would still need to be recapitulated and confirmed in an independent patient cohort before being further developed as a clinical diagnostic.
Also, many more mechanistic experiments and longitudinal retrospective cohort analyses would need to be done to confirm their association with aspiration-related respiratory infections.
In Chapter 3, the consistent associations that I identify across many different gut microbiome studies and microbiome-related diseases can make no claim of causality: in all case-control microbiome datasets, it is not possible to determine whether the microbiome shifts are a cause of the disease or simply responding to the host's physiological state.
In fact, finding many non-specific disease- and health-associated bacteria across these datasets suggests that a large part of these associations are more likely to be responses to the host's general health status rather than specifically causal to any given disease.
To truly find disease-associated bacteria, researchers need to identify consistent associations across multiple cohorts of their disease of interest which are also specific to that disease, and isolate the strains of interest to test their and specificity in animal models.
Finally, the conceptual framework that I present in Chapter 4 suggests one way to advance translational microbiome science but will need to be applied in many FMT trials before its impact on FMT trial successes can be confirmed.

A second limitation comes from the fact that these projects are anchored in the analysis of 16S rRNA data, which is more accessible than many 'omics data types but comes with its own set of inherent limitations.
16S data provides a window into which bacteria are present in microbial communities but does not indicate what functions these bacteria are performing.
Because disease-mediating effects will result from bacterial function, future work should strive to incorporate function into their studies, for example by analyzing metagenomics, metabolomics or transcriptomics data, or a combination of these data types.
I hope that as datasets become more readily available,  someone will perform a similar cross-disease meta-analysis to mine, but based on metagenomics data.
I expect that a function-focused meta-analysis will find much more consistent and readily interpretable associations within studies of the same disease and across diseases.
Another limitation of 16S data is that in most cases it cannot resolve bacteria at the strain level.
Given that different strains of the same species have dramatically different clinical presentations, this is a crucial limitation in extracting clinically relevant associations from 16S microbiome data.
Strain-level resolution will be especially important in evaluating rational donor selection methods, where engraftment of specific donor strains may be an important factor mediating different patient responses to FMT.

Finally, these studies are limited by small sample sizes.
Although the aerodigestive cohort presented in Chapter 2 and the meta-analysis in Chapter 3 are the largest of their kind to date, the analyses that I could perform were still limited by their sample sizes.
In Chapter 2, I was unable to draw robust conclusions about the relationship between reflux and the aerodigestive microbiome because the number of patients with the respective samples and metadata was too small to sufficiently power my analysis.
In fact, throughout this project I frequently found my analyses limited by the necessary confluence of samples and metadata.
For example, I would have liked to analyze the combinatorial effects of proton pump inhibitors, aspiration, and reflux on the aerodigestive microbiome, but I simply did not have enough patients with the respective samples and metadata to perform any reasonably powered analyses.
The meta-analysis presented in Chapter 3 was also limited by the number of datasets and diseases.
Originally, I had hoped to compare disease-associated microbiome shifts in an unsupervised manner to identify patterns of shifts common to similar types of diseases.
For example, I wondered if I could find a consistent group of bacteria associated with inflammation across multiple inflammatory diseases.
However, I had a surprisingly difficult time finding datasets with publicly available raw data and associated patient-level clinical metadata, even given the large corpus of relevant published papers.
Thus, I did not acquire a large enough variety of diseases and datasets to enable this broader categorization of diseases and perform statistically meaningful analyses on these groups of datasets.
Finally, one of the main conclusions of the framework presented in Chapter 4 is that the usual sample sizes in FMT studies will almost always limit the power of retrospective analyses to identify key bacteria.
Because these studies are performed with human samples and sometimes require careful ethical justifications and considerable patient recruitment efforts, this challenge of small sample sizes is likely to remain an important limitation for future studies throughout the microbiome field.

\section{Perspectives and future work}

%Maybe project-specific extensions? If I had infinite time and money/questions I wish I had the answer to:

%Aerodigestive: why is stomach-lung so similar within patinets? Is it stochastic, is it that the majority of bacteria captured by 16S are just dead and being cleared out and therefore irrelevant?

%Meta-analysis: do it with function - do you see consistent results? Dig into the question of core health and disease: can we use disease as a perturbation to identify the core components of a health microbiome?

%FMT framework: simply put, I want to know if it works. Hopefully clinicians will start to use rational donor selection, and enough variability in study designs and outcomes will exist such that in a few years we can look back and see if choosing donors rationally led to any difference in patient outcomes and/or if it helped us learn more about microbiome's role in any diseases.

\subsection{Re-analyzing existing datasets and data availability}

A core theme that has emerged from my thesis work is that re-analyzing existing microbiome datasets has the potential to add substantial value to our field.
One of my personal conclusions from the meta-analysis is that cross-sectional gut microbiome studies should be regularly contextualized within the existing published body of work.
Now that microbiome datasets are more readily available and bioinformatics tools to process and analyze them are becoming very accessible, I hope that researchers make it a habit to ask: are these these results specific to my disease of interest? Are these associations part of a non-specific response to health and disease? Are these associations consistent with those from independent patient cohorts?

I also found the value of re-analyzing datasets especially relevant in research areas led by clinicians.
For example, the original IBD FMT trials that I re-analyzed in Chapter 4 did not thoroughly investigate potential donor effects related to patient response in their data.
This is expected in part because the question of heterogenous donor response was not usually their primary question of interest, and also in part because their trials were not powered for these analyses.
However, now that multiple IBD FMT trials have been published with patient and donor microbiome sequencing data, hypotheses about what leads to improved patient response could be tested \textit{in silico}, as we did for butyrate producer abundance in Chapter 4.
% maybe: Finally, re-analyzing published datasets adds value at little extra cost: recruiting patients, collecting samples, and generating data

Moving forward, I hope and expect that researchers continue to make their raw data publicly available and that re-analyses of such data become standard in the field.
Publicly available raw data allows researchers to ask and answer new questions, testing their hypotheses \textit{in silico} without the need for costly new studies.
Also, new studies can compare their results with existing work to determine which of their findings hold across different studies and which are specific to their specific study.
However, using published data in new contexts comes with its own challenges.
Batch effects, which result from different experimental and sequencing methods, can make it very difficult to compare data across different labs.
I contributed to a method to correct for batch effects in case-control studies \cite{gibbons-2018}, and I hope that batch-correction methods keep being developed for this data type.
Another major challenge is data availability, specifically metadata: raw data without its associated metadata is in almost all cases useless.
Standards for sharing data which respect patient privacy, clinicians' efforts for patient recruitment, and the needs of computational biologists re-analyzing these data will need to be agreed upon and upheld as a community.

\subsection{Partnerships between clinicians and computational biologists}

Throughout this thesis, I also came to appreciate the unique contributions that come from close partnerships between clinicians and computational biologists.
None of the projects in this thesis would have been possible without crucial contributions from our clinical counterparts.
Chapter 2 was only possible because our clinical PI, Rachel Rosen, identified and framed the questions of interest in this cohort so that I could then translate them into computational analyses that could provide answers with our data.
Working together in this way, we discovered new science and found clinically exciting results.
The meta-analysis in Chapter 3 was also significantly strengthened by the inclusion of datasets which were originally purely clinical investigations, and the framework presented in Chapter 4 was only possible because of our lab's strong ties with the clinical experts at OpenBiome.
I hope that the future of translational microbiome research establishes structures that encourage such close collaborations.
Systems to share raw data should be designed with these collaborations in mind: the process to deposit data should be accessible to clinicians, patient information should be protected while also providing easy access to analyses that don't use the protected information, and the metadata should be curated well enough to enable straightforward analyses without much manual curation but also flexible enough to allow for the variety of study designs pursued by clinicians.

\subsection{Finding knowledge in the information}

A turning point in my thesis came when I read Gene Glass's 1976 paper coining the term "meta-analysis" \cite{glass-1976}.
In it, he argues for the importance of consolidating and synthesizing information into knowledge, prizing work that aims to find meaning and draw conclusions from existing studies.
This was a turning point for two reasons.
First, it was the moment I became truly proud of my work, especially Chapter 3's meta-analysis: I felt validated in the difficulty and value of the work I was doing and no longer dismissed it as merely "stamp collection" work that anyone with basic knowledge of computational tools could do.
Second, it helped me realize that a uniting theme in all of my work was exactly this, finding the meaning from all of the information in this data and this field.
I realized that each project in this thesis, in its own way, aimed to move beyond statistical associations and was not satisfactory until it went all the way to its clinical implications.
I hope that as we move forward, the field of microbiome research becomes less satisfied with reporting information and continues to strive for synthesizing the knowledge and clinical impact.


\section{More words}

Sample sizes: even though aerodigestive in the largest cohort, it's still too small to make robust inferences about things of interest (like reflux). Even though 30 datasets is a lot, it's nowhere near the size of "real" meta-analyses. And, as we show in donor selection chapter, sample size will continue to be a limitation in trying to extract new discoveries from FMT trials.

Associations not causation: in all three projects, I only show associations. Need to go back and validate with larger patient cohorts and orthogonal confirmations (mechanistic model organism studies, randomized trials, culturing bugs to check that they matter, etc)

Taxonomy but not function-based: in all three projects, I mostly focus on what we can learn from 16S data. But function is probably way more relevant: need to look at metagenomics or transcriptomics. And also, strains are really the thing that matters in this data. For FMT stuff especially, will need robust ways to track strains in complex communities. Also meta-analysis, need to confirm what the bugs are doing.

\subsection{Extensions}

Aerodigestive: ??

Meta-analysis: discuss data standardization, difficulty of accessing *metadata* (which nothing, even Qiita, addresses). Also batch correction methods.

FMT: need better ways to identify individual microbes. Go back to culturing (basically, talk myself out of a job). Ask better questions at the outset of a clinical trial.

This thesis describes contributions to the interpretation of microbial
ecology sequence data and to the design of clinical trials. These contributions
each have limitations that restrict their validity and applicability.


\subsection{Partnerships between clinicians and comp biologists}

Systems to deposit raw data which are accessible to clinicians, protect patient information, and enable

structures and support that consider the needs of both communities (need an easy way to deposit data for it to not be a barrier to clinicians, database that protects patient information but does not prohibit simple analyses that don't use the protected patient information [give example of my data for Chapter 2], and needs to be organized/curated well enough that there doesn't need to be too much manual work for the analyst)

Although working closely with clinicians can be challenging, it is ultimately the best way to impactful translational research.
As I learned in Chapter 2, it is important to


Despite some of the challenges that come with attempting to bridge clinical and basic science research, this is really the best way to do translational work.



However, working closely with clinicians can be difficult because our perspectives on and goals for the microbiome research are not always perfectly aligned.
As an example from Chapter 2, I was most excited about the scientific finding that lung and stomach microbiomes are driven by people rather than body site, but Rachel was most excited about the clinical finding.
Thus, writing these results

, Rachel Rosen's
The aerodigestive analysis presented in Chapter 2 required my clinical collaborator, Rachel Rosen, to

Something else that I learned to appreciate in the course of this thesis is

Now that high-throughput sequencing is cheap and accessible enough to be incorporated in many clinical studies


Opportunities: literally every project in this thesis
First project literally couldn't be done without Rachel.
Second project: included datasets from clinical studies and research studies, made the paper much stronger! (Also much harder, lulz)
Third project: illustrates the power of putting the two expertises together, and was only made possible by our lab's strong ties with the clinical experts at OpenBiome.

Challenges (mostly from Chapter 2):
- dramatically different goals and scientific questions (give example of aerodigestive: I found super interesting science, she found super interesting clinical. Hard to put together in one story.)
- really different priorities: goal in clinical spaces may be to find *anything* that works, whereas I found myself uncomfortable reporting anything except that which was absolutely supported by the data
- lack of skills/knowledge means that we don't do a good job of bridging the gaps between our fields of expertise: many bioinformatics tools are not useable to clinicians, much of the data deposited by clinicians is not useable to computational biologists

Opportunities:
- collaborations and partnerships
- incentives that reward clinicians for sharing valuable clinical data
- structures and support that consider the needs of both communities (need an easy way to deposit data for it to not be a barrier to clinicians, database that protects patient information but does not prohibit simple analyses that don't use the protected patient information [give example of my data for Chapter 2], and needs to be organized/curated well enough that there doesn't need to be too much manual work for the analyst)

\subsection{knowledge from information}

strives for clinical impact

I hope that coming generations of microbiome scientists

First project: have to go beyond just looking for statistical significance and try to figure out what would be clinically meaningful.

Second project: just listing bugs doesn't mean anything if you don't contextualize them. More work should be done to identify a generic "healthy" and "disease" microbiome.

Third project: need to incorporate your a priori knowledge into everything you do. Have to think about *why* a trial would fail, and whether or not you think that's legit or not. Also, remember to try very hard to ask better questions at the outset of a clinical trial.

A turning point in my thesis came when I finally read Gene Glass's paper that coined the term meta-analysis: his words validated the importance and difficulty of synthesizing published data to extract meaningful knowledge from it.


\subsection{scott's stuff}

In Chapter 2, I introduced \texttt{texmex}, a tool designed to quantify the dynamics of
microbial taxa in microbial ecology experiments that use amplicon sequence
data, use pre-tests, and have few or no replicates. I expect this approach
will be helpful when researchers want to analyze a pilot experiment, the
environmental inoculum is difficult to acquire, or the experimentation is
particularly onerous. Ideally, a researcher would perform many replicate
experiments and use that information for a rigorous statistical inference that
does not require any special consideration of the ecological structure of the
data in question, thus obviating the need for a technique like \texttt{texmex}.
Because the method is not statistical, however, it will never supplant
methods that are designed to determine whether two sets of measurements are
meaningfully different from a statistical standpoint.

In Chapter 3, I introduced the operational ecological unit (OEU) and
the inferred biomass interpretative framework to link taxonomic survey data, an
ecosystem-level metabolic model, and the results from a single-cell genetic
assay. The model itself is conceptual and intentionally simple; it therefore
lacks the ability to describe complicated features of the lake
ecosystem it models. For example, the model could never predict the
hypolimnetic oxygen minimum observed in the survey data. Operational
ecological units are essentially statistical and not necessarily functional,
so it is not straightforward to confirm or disprove their ``existence''. The
utility of the OEU concept could be evaluated by comparing an analysis of OEUs
with a large database of known ecological interactions. The inferred biomass
framework makes concrete, verifiable claims about microbial community function
that could be compared with metagenomic data and, ultimately, verified or
disproved by comparison with an exhaustive, \textit{in situ} survey of
ecosystem function. The results of the study are overall very suggestive, but
they are not experimentally verified and would require extensive co-culturing
or perturbative, \textit{in situ} metabolic experiments to validate.

In Chapter 4, I introduced a model of differences in donors' stool with
respect to its probability to cause patients to respond to treatment with fecal microbiota
transplant. The model makes concrete predictions about the utility of clinical
trial designs, but the structure of the model is based on a small amount of
clinical trial data and would require extensive clinical trial data to verify.
This study is in a catch-22: it aims to improve the probability of finding the
statistically-significant clinical trial data that would provide the only way to
assess the validity of the model.

\section{Potential extensions of reported work}
\subsection{Rank-abundance distributions and small data}
In a narrow sense, \texttt{texmex} is a software package that converts OTU tables
into tables of values related to the initial counts and provides convenience
functions for manipulating and selecting interesting OTUs based on those
transformed values. More broadly, that work makes two contributions that should
be helpful for future efforts to improve the interpretability of DNA sequence
data in microbial ecology.

First, I was surprised that I could find no studies that directly
examined or utilized the rank-abundance distribution of microbial ecology
sequence data. (I found only one paper that fit a rank-abundance distribution to
microbial ecology data~\cite{kembel_incorporating_2012}, and I describe what I
perceive as deficiencies in the logic of its application in Chapter 2.) I
believe that there are many applications that will emerge from using such
rank-abundance distributions, just as $z$-scores make normally-distributed data
more tractable for analysis. In particular, I expect that any attempt to
compare OTUs across samples could benefit from the kind of ``normalization''
that \texttt{texmex} does.
The standard statistical approach, in which an OTU's counts across samples are
modeled as variates of a single random variable, seems like a weak approach
compared to focusing on the ecological processes that cause the entire community
to assemble.

This sort of sample-wise approach should also be helpful for understanding some
of the more confusing aspects of microbiome data, particularly the zeros and
the effects of rarefaction.
It is becoming clearer that micro- and macroecology can share their methods~\cite{hughes_counting_2001},
so microbial ecologists should, in some cases, pay closer attention to the
methods and approaches used in traditional ecology.

Second, \texttt{texmex} starts from a very different place from many other
analytical methods: it assumes a paucity of data rather than an abundance.
As DNA sequencing has become cheaper, it is tempting to believe that microbial
ecology is now limited only by the cleverness of the algorithms used to generate
the data or the cleverness of the scientists who decide what questions to
investigate. In fact, sample acquisition is still, in many cases, a limiting
factor in microbial ecology, as has been my experience in the project described
in Chapter 2 (as well as in a separate project in coordination with the Department
of Energy).
If a study is not comfortably in the regime of
big data, I believe it is wiser to relegate yourself to the regime of small
data. Although you can ``do statistics'' with three samples, if you cannot
get twenty samples, it might be wiser to collect two and use a small-data
technique for the first experiment. As reviewers of the manuscript pointed
out, it is always better, \textit{ceteris paribus}, to have more replicates.
My point is that having more replicates always come at some cost, and the
added replicates might deliver a $p$-value without any additional scientific insight.

I was impressed that a recent paper studying oil degradation pathways in
samples collected from the Deepwater Horizon spill---and which used an impressive
combination of isotope labeling and metagenomic sequencing---identified similar organisms
as my algorithm did~\cite{dombrowski_reconstructing_2016}, showing the power
that small data and wise analytics can have.

There were extensions of this work that were outside the scope of this
thesis. Are there datasets that are sufficiently well-resolved to be able
to distinguish the rank-abundance distribution of microbial ecosystems?
Is that distribution the Poisson-lognormal distribution or something else? Is it different
for different ecosystems? What does that tell us, theoretically,
about the structure of those communities? Can we use rank-abundance models
to avoid the problems that compositional data present for analysis?
Can we use rank-abundance distributions to infer more rational models of
the behavior of taxa across samples? Relatedly, can we use rank-abundance
models and timeseries data to draw inferences about the dynamical behaviors
of individual taxa and entire bacterial communities? Can we better explain
the overdisperse and apparently noisy behavior of taxa through time?

\subsection{Modeling, consortia, and combinations of methodologies}
Like \texttt{texmex},
the methods used for the project described in Chapter 3 also aimed to extract the maximum amount of
insight from limited data. This project integrated the results from
multiple methods to yield a single, biologically-interpretable discovery.
This integration carries some lessons of its own.

First, models of microbial communities should aim for an optimum between complexity
and falsifiability. Because the data generated by DNA sequencing are so
massive and so complicated, it is tempting to make a complicated model
of their behavior. However, even if such a model were made and even if it
correctly recapitulated the system's behavior, are we better off for
having it? For example, the model reported in this project used abstract
categories (e.g., sulfate reduction) to describe microbes' behavior.
The identity of the microbes that seemed to belong in that abstract
category was determined separately, and it is the link between the
microbes' identity and the abstract behavior in the model that was useful.
If the model had perfectly described the behavior of every microbial
species, then we would have produced a system exactly as complicated
as the natural one, which would not advance our ability \emph{interpret}
the system.

Indeed, one of the strengths of the model presented in that work is that
it was \textit{a priori} unclear if it would even remotely recapitulate the lake's chemical
dynamics. If it did not, then it would be immediately clear that our mental
picture of the processes that shape the lake's behavior was largely
incomplete. Adding another process into the model (e.g., the interaction
between iron and sulfur) would hold the model and the associated data
to a much more stringent measure: it would require a great enough precision
in the data to be able to distinguish between the cases in which the
iron-sulfur interactions are included or not. Given the year-to-year
variability in this system, an ecosystem-wide model is not the appropriate
tool for making that discrimination. The simple fact that the model
worked at all---and that, if it had not worked, we would have learned
something too---is its major contribution. A model that is not interesting
if it fails is one that should not be considered interesting if it
succeeds.

There is probably great opportunity for modeling in microbial systems.
The model used in this project was based on a pre-next-generation-sequencing
model of microbes in a groundwater aquifer responding to pollutants,
which also highlights the fact that literature from before 1990 can
be full of interesting insights and thorny questions that we now have
the tools to explore more deeply.

For example, despite direct measurements of bacterial growth in
zebrafish~\cite{jemielita_spatial_2014} and a bacterium genetically
engineered to answer questions about the rate of division in the gut~\cite{myrhvold_distributed_2015},
I have not seen a model of division and colonization in the mammalian
gut that accounts for its directionality: how does the unidirectional transport
of vast quantities of microbes from the ``top'' to the ``bottom'' affect
the microbial composition in the gut? Are downstream populations
less diverse than upstream populations because every microbial species
present at the bottom must have once been near the top?

Second, this project makes an early, unrefined estimate of the prevalence
of microbial consortia in natural environments. Before nucleotide sequencing,
bacterial species were distinguished based on their appearances or tests
of their metabolisms. This process was finicky and low throughput, so we
had vastly underestimated the diversity of microbes. I believe we are on
the cusp of a similar revelation about microbial consortia. I expect that
theoretical arguments would show that a large number of cooperative species
should be expected, and this contrasts against the very small number of
consortia that are known and studied. The possibility that there are
large numbers of consortia in many ecosystems is probably the most
scientifically interesting and important result in this thesis.

Third, this project shows the potential that combinations of methods
can play in understanding microbial systems. Surveys on their own
do little to address microbial function; models on their own
can seem like intellectual playgrounds unconnected to reality; high-throughput
screens on their own can generate large amounts of data with small amounts
of insight. In particular, I expect that combinations of models, surveys,
and metabolite measurements will provide interesting and useful
information about the interactions between microbial species (and hosts if
they have them).

\subsection{Decision-making in microbiome science}
The third project in this thesis is an outlier: it describes a
simple model---like Chapter 3 does---but it uses the model for
an entirely different purpose. Rather than developing information
about the possible behavior of a system, it uses a model and data to
make a decision in face of a question. (This distinction is reminiscent
of the difference in interpretations of the $p$-value~\cite{goodman_toward_1999} between Fisher,
who originally formulated it as a method to discern truth~\cite{fisher_statistical_1973}
and Neyman and Pearson, who saw it as a way to decide actions~\cite{neyman_problem_1933}.)
I will venture to say that most models in the world are, like this one,
operational models: they are designed to integrate data to inform a
decision.

DNA sequencing is already being used in medicine to, for example,
diagnose infections, and there is hope that more sophisticated,
rapid, point-of-care diagnostics will be useful to, say, use information
about the genetics of the pathogen to decide
which antibiotic to administer to a patient.
However, the role of modeling in decision science
for microbiome science, as such, remains unclear. In what cases could
a large collection of information about the microbes inhabiting a
person's gut be useful for making a decision? What decision would
be made?

There are some appealing answers. Measurements of the microbiome
could be used to diagnose a disease that is otherwise difficult or
invasive to diagnose \cite{papa-noninvasive-2012}, to quantify the
risk that a patient will develop a disease, or to help stratify
patients based on the probability that they will respond to certain
drugs \cite{koeth-intestinal-2013,sivan-commensal-2015,vetizou-anticancer-2015}.
I expect a ``middle'' way will also be profitable. A model that
combines a simple treatment of a system (e.g., as in Chapter 4,
each donor is considered efficacious or not) and a more complex one
(e.g., it is asserted that the presence or absence of some microbial
taxon in the donor determines the probability of patient response)
could recommend decisions that are nearly optimal with respect to
the simple, operational model while deriving greater benefit for
the more complex, mechanistic model. This operational half of
the approach might get complex hypothesis-testing into the clinic,
since the simple half of the algorithm could be relied upon to
make sensible decisions even if it became clear that the complex,
mechanistic model was completely incorrect.

In general, I caution microbiome scientists against interpreting too
much from 16S sequencing data. The fact that DNA sequencing is a
less-biased way to enumerate communities than traditional
culture-based methodologies may have reduced the emphasis on the
problems that DNA sequencing presents: the microbiome appears to
be a dynamic, noisy system; extraction and preparation methodologies
greatly affect the output signal; different bioinformatic techniques
can lead to different scientific conclusions; and proper methods
of statistical analysis for these data are still under debate.
Targeted questions with large sample sizes and perturbative
techniques are the best avenue for conclusions; small experiments
with decidedly exploratory analytical methods are the best
avenue for developing avenues for fuller investigation.

\begin{singlespace}
\bibliography{main}
\bibliographystyle{unsrt}
\end{singlespace}
