%% This is an example first chapter.  You should put chapter/appendix that you
%% write into a separate file, and add a line \include{yourfilename} to
%% main.tex, where `yourfilename.tex' is the name of the chapter/appendix file.
%% You can process specific files by typing their names in at the
%% \files=
%% prompt when you run the file main.tex through LaTeX.

\chapter{Introduction}

\section{Human-associated microbial communities are essential for health and implicated in disease}

Intro paragraph about the human microbiome relevance in health and disease. Make sure this paragraph is site-agnostic.

Motivate why we're excited about using the microbiome for therapeutic applications - written out version of my usual talk intro.

Discuss big data aspect of microbiome research: sequencing provides opportunities and challenges. In this thesis, we present three projects which overcome these challenges to provide useful clinical insight from mining microbiome data.

From Sarah's: "In this thesis, I present three studies of bacterial functional and spatial structure that highlight innovative new molecular biology techniques and methods to increase throughput by reducing reaction volume. The following chapters each contain a unique application in environmental or human microbial communities that show the critical need to link genes to hosts, and link hosts to local community members and microenvironments."

\section{Multi-site sampling of the aerodigestive microbiome allows for the identification of clinical associations}

Introduce aerodigestive microbiome. Discuss challenges: sampling is invasive so it's not very well-studied. Lungs communities especially are very variable, making it difficult to identify specific biomarkers.

However, identifying microbial biomarkers would be useful: these patients are already undergoing invasive tests, and reducing radiation would be good.

In Chapter 2, we present an analysis that leverages simultaneous sampling of patients to identify clinically-relevant characteristics distinguishing patients with swallowing dysfunction from those with normal swallow.

Mention that this patient population is already undergoing these invasive tests, so a microbiome-based alternative would actually be useful.

We overcome the challenge presented by huge variability between lungs by looking within patients.

Suggest new avenues for approaches to treating aspiration.

\section{Re-analyzing data across many diseases identifies disease-specific and shared patterns of associations between the gut microbiome and disease}

In contrast to aerodigestive, gut microbiome has been extensively studied. Many diseases associated with gut microbiome, and healthy functioning is essential for host health. However, lots of research but little consensus on which associations are clinically meangingful.

Challenges to synthesizing findings: different data processing, different analyses. Clinical studies and scientific studies have different goals and bioinformatics/analysis approaches.

In Chapter 3, I perform a cross-disease meta-analysis from raw data.

\section{Translating microbiome research through rationally designed fecal microbiota transplant clinical trials}

Important way to move from associations to clinical impact is through FMT. In model organisms, FMT is useful to prove causality and test mechanism. In humans, FMTs are useful to identify promising clinical areas where microbiome might be implicated.

As our understanding of microbiome associations with diseases increases, FMTs will be used to drive the development of therapies. Coupled with exitence of stool banks, we can use what we learn to do better FMT trials.

In Chapter 4, I present a framework for rational donor selection in FMT clinical trials, where the hypothesized type of microbiome association drives the way you do your trial.

This chapter talks about what to do once you have figured out how your disease is being mediated by the microbiome.

\begin{singlespace}
\bibliography{main}
\bibliographystyle{unsrt}
\end{singlespace}
