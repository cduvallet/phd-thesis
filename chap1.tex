%% This is an example first chapter.  You should put chapter/appendix that you
%% write into a separate file, and add a line \include{yourfilename} to
%% main.tex, where `yourfilename.tex' is the name of the chapter/appendix file.
%% You can process specific files by typing their names in at the
%% \files=
%% prompt when you run the file main.tex through LaTeX.

\chapter{Introduction}

\section{The human microbiome in health and disease}

The human microbiome, made up of the microbes that live in and on our bodies, is essential for health and has been implicated in many diseases.
Almost all human body sites are colonized by microbial communities, ranging from the gut, the largest human-associated microbial community, to the lungs, which were for many years considered sterile (citation needed).
In health, these communities fight off or prevent infections, interact with and regulate host immune responses and metabolism, and metabolize xenobiotics or other compounds which are indigestible by the host, among other functions.
Microbes can also cause or exacerbate disease, as well as impact how we metabolize the drugs we take.
Perturbations in the composition or function of these communities have been implicated in many diseases, including metabolic, autoimmune, neurological, and respiratory conditions.
<Maybe a sentence about therapies?>
Although humans known about the human microbiome for thousands of years, recent academic research into this field has expanded significantly (citation needed).
Emerging research has been driven in part by the increasing accessibility of DNA sequencing technology, which can provide a detailed view of the bacteria in these communities without the need for time-consuming and difficult culturing experiments.

Excitement in this field from scientists, engineers, clinicians, and entrepreneurs has also been driven by the non-invasiveness of the research, large range and accessibility of potential interventions, and the increasing potential for clinical impact.
Many human-associated communities can be sampled non-invasively, facilitating research that is performed with real human patient populations rather than model organisms.
Interventions which can change the composition of microbial communities already exist and are accessible to patients: lifestyle changes such as diet can have large effects on the microbiome, and commonly used drugs like antibiotics and probiotics are already in wide usage.
Thus, identifying ways to repurpose existing interventions for targeted outcomes is attractive and exciting because the potential to implement them quickly is high.
Finally, the potential clinical impact of microbiome-based interventions is vast and growing.
Fecal microbiota transplants (FMT) for recurrent Clostridium infection have already proven themselves as an effective treatment: over 80% of patients with recurrent C. diff are cured after FMT.
Furthermore, many of the diseases in which the microbiome may play a role are associated with industrialization and are increasing in incidence worldwide.

Despite the increasing accessibility of microbiome research and potential for clinical impact, identifying clinically-relevant associations in microbiome studies remains challenging.
Microbiome datasets are high-dimensional, containing hundreds to thousands of bacterial species but frequently only sampling tens to hundreds of patients.
Microbial communities are also highly variable across people, and identifying  individual bacterial biomarkers that can distinguish health and disease across many patients is difficult.
Finally, microbiome datasets provide a window into associations but not causal relationships.
Researchers must perform follow-up mechanistic studies or FMT clinical trials to confirm the clinical relevance of associations.
In this thesis, we present three projects which overcome various data analysis challenges to provide useful clinical insight from mining microbiome datasets.
The following chapters each present a unique analysis of microbiome data, each highlighting a different perspective on harnessing microbiome data to move beyond statistical significance and identify clinically meaningful associations.

\section{Multi-site sampling to identify clinical associations in the aerodigestive microbiome}

In Chapter 2, we present an analysis that leverages simultaneous sampling within patients to identify clinically-relevant aerodigestive microbiome characteristics distinguishing patients with swallowing dysfunction from those with normal swallow.
Currently, diagnosing aspiration resulting from impaired swallow function involves imaging a patient as they ingest barium-labeled food or liquid.
Identifying lung microbial biomarkers would provide a useful non-radioactive alternative to this diagnosis.
Additionally, patients with impaired swallow function are at higher risk for respiratory infections, but the extent to which the lung microbiome is perturbed by aspiration and thus potentially involved in mediating this risk is unknown.
Finally, clinical interventions to treat respiratory symptoms in patients with impaired swallow focus on preventing material transfer from the stomach into the lungs, for example via anti-reflux medication or fundoplication surgery.
However, the clinical utility of these interventions in pediatric populations is not fully established, and the extent to which the gastric vs. mouth microbiome mediates respiratory complications is also not known.

In this study, I analyze a set of aerodigestive microbiome samples collected from over 200 patients at Boston Children's Hospital.
I first show that lung and stomach microbiomes are highly variable across people and driven primarily by person rather than body site, thus complicating the search for a reliable microbial biomarker of aspiration across people.
I overcome this challenge by leveraging the fact that we have multiple samples per patient, analyzing differences in the within-patient _relationships_ between microbial communities across the aerodigestive tract.
Using this approach, I show that aspiration shifts lung microbial communities toward the oropharyngeal but not the stomach microbiome, suggesting that the mouth is likely an important source of microbes and perturbations in the lung communities of these patients.
Thus, approaches for treating aspiration-related respiratory symptoms should target microbial transfer from the mouth into the lungs in addition to focusing on the gastric-lung axis as most current interventions do.
This study also illustrates the ability of multi-site within-patient sampling to overcome variability across people to identify clinically meaningful microbiome-base biomarkers.

\section{Re-analyzing datasets identifies consistent patterns of associations between the gut microbiome and disease}

In contrast to aerodigestive, gut microbiome has been extensively studied. Many diseases associated with gut microbiome, and healthy functioning is essential for host health. However, lots of research but little consensus on which associations are clinically meangingful.

Challenges to synthesizing findings: different data processing, different analyses. Clinical studies and scientific studies have different goals and bioinformatics/analysis approaches.

In Chapter 3, I perform a cross-disease meta-analysis from raw data.

\section{Translating microbiome research through rationally designed fecal microbiota transplant clinical trials}

Important way to move from associations to clinical impact is through FMT. In model organisms, FMT is useful to prove causality and test mechanism. In humans, FMTs are useful to identify promising clinical areas where microbiome might be implicated.

As our understanding of microbiome associations with diseases increases, FMTs will be used to drive the development of therapies. Coupled with exitence of stool banks, we can use what we learn to do better FMT trials.

In Chapter 4, I present a framework for rational donor selection in FMT clinical trials, where the hypothesized type of microbiome association drives the way you do your trial.

This chapter talks about what to do once you have figured out how your disease is being mediated by the microbiome.

\section{More words}

Identifying modifications in these drugs

Interventions which can change the composition of microbial communities are also relatively accessible: diet and lifestyle are known to have large effects on the human microbiome, and existing drugs like antibiotics and probiotics are already in wide usage.

Many possible ways to intervene to improve microbiome: lifestyle changes, antibiotics, probiotics

There is also an exciting possibility of repurposing existing interventions like antibiotics and probiotics for greater or novel therapeutic effects.

There is also strong precedent for microbial-based therapies, and the

allowing for studies to iterate between real human patients and model organisms

Motivate why we're excited about using the microbiome for therapeutic applications - written out version of my usual talk intro.

with the motivating goal

Discuss big data aspect of microbiome research: sequencing provides opportunities and challenges.

From Sarah's: "In this thesis, I present three studies of bacterial functional and spatial structure that highlight innovative new molecular biology techniques and methods to increase throughput by reducing reaction volume. The following chapters each contain a unique application in environmental or human microbial communities that show the critical need to link genes to hosts, and link hosts to local community members and microenvironments."

-----

Furthermore, this analysis suggests that future studies investigating clinical associations in lung and stomach microbiomes

Suggest new avenues for approaches to treating aspiration and for analyzin future lung microbiome studies.

Mention that this patient population is already undergoing these invasive tests, so a microbiome-based alternative would actually be useful.

We overcome the challenge presented by huge variability between lungs by looking within patients.

Patients with impaired swallow function are at higher risk for respiratory infections, but the mechanism by which these infections are mediated is unknown.
Clinical interventions like fundoplication surgery and anti-reflux medication attempt to reduce transfer of material and bacteria from the stomach to the lungs, which is thought to drive respiratory complications, but often do not show clinical benefit.

Additionally, the extent to which the microbiome is perturbed by and perhaps implicated in aspiration-related respiratory complications is unknown.

Further complicating this is that we lack a comprehensive understanding of hte lung and stomach microbial communities: these samples require invasive sampling procedures and are thus difficult to acquire, especially in health people.
In fact, neither the lung nor stomach were included in the first iteration of the Human Microbiome Project, and the lung was thought to be sterile for many years.

The aerodigestive tract is comprised of the upper and lower respiratory tracts and the upper gastrointestinal tract, including body sites such as the oropharynx, lungs, and stomach.
Unlike most other body systems, material in the aerodigestive tract flows between body sites bidirectionally.
For example, swallowing carries material from the mouth into the stomach, but vomiting or reflux can also lead to exchange in the opposite direction.
Thus, microbial communities in this body system are shaped by a balance of immigration, elimination, and growth (Dickson Lancet 2014).
However, compared to other human-associated communities like the gut and skin microbiome, relatively little is known about the communities in the lung and stomach.
In fact, the lungs were long thought to be sterile and neither the lung nor stomach were included in the first iteration of the Human Microbiome Project.
Despite the emerging recognition that these sites are also characterized by microbial communities which perform important functions, it remains difficult to study these in part because acquiring lung and gastric microbiome samples requires invasive procedures such as bronchoscopies and upper endoscopies.
Thus, we lack a baseline understanding of these communities in health.

Introduce aerodigestive microbiome. Discuss challenges: sampling is invasive so it's not very well-studied. Lungs communities especially are very variable, making it difficult to identify specific biomarkers.

However, identifying microbial biomarkers would be useful: these patients are already undergoing invasive tests, and reducing radiation would be good.

In Chapter 2, we present an analysis that leverages simultaneous sampling of patients to identify clinically-relevant microbiome characteristics distinguishing patients with swallowing dysfunction from those with normal swallow.

Mention that this patient population is already undergoing these invasive tests, so a microbiome-based alternative would actually be useful.

We overcome the challenge presented by huge variability between lungs by looking within patients.

Suggest new avenues for approaches to treating aspiration and for analyzin future lung microbiome studies.



\begin{singlespace}
\bibliography{main}
\bibliographystyle{unsrt}
\end{singlespace}
